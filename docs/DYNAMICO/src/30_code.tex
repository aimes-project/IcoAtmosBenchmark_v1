\section{Overview and common stuff}

\subsection{Kernelize}\label{s:kernelize}


Kernel programs in this package are as follows;
%
\begin{itemize}
 \item \src{comp_pvort}
 \item \src{comp_geopot}
 \item \src{comp_caldyn_horiz}
 \item \src{comp_caldyn_vert}
\end{itemize}
%
All kernels are single subroutines in the original\footnotemark
\DYNAMICO, and extracted and imposed to the wrapper for the kernel program
%
Each subroutine has no modification, except using modules and some
parameter settings.


All of input arrays and most of arrays/variables defined in various
modules in the original model are read from input data file.
%
Input data for each subroutine and reference (output) data are dumped
from the execution of original \DYNAMICO.
%
Main routine of each kernel program reads these input and reference
data, and call the subroutine with them as arguments for 1000 times, in
current setting, then compare output values with reference data.


\footnotetext{
Note that ``original'' here means ``before kernelize'',
since some bug fixes have made by AICS.
Please contact the address shown on the back cover of this manual for details.
}


Kernel programs output several log messages to the standard output, such as:
\begin{itemize}
%\setlength{\itemsep}{0pt}
 \item min/max/sum of input data,
 \item min/max/sum of output data,
 \item min/max/sum of difference between output and validation data,
 \item computational time (elapsed time).
\end{itemize}
%
Elapsed time is measured using \src{omp_get_wtime()}.

There are sample output files for the reference in \src{reference/} directory
of each kernel program, and also they are shown in``Input data and
result'' section of each kernel program in this document.


\subsection{MPI and OpenMP}

While original \DYNAMICO is parallelized by MPI and OpenMP, all kernel
programs in this package are meant to be executed as one process with
no threading.

Different from the \NICAM kernel programss in this package, you don't
need MPI library to compile/execute \DYNAMICO kernel programs, but you
need to make OpenMP enable in order to use \src{omp_get_wtime()}.

\subsection{Mesuring environment}\label{s:measuring_env}

In the following sections, the example of performance result part of the
log output file of each kernel program is shown.
%
These were measured on the machine environment shown in
\autoref{t:machine_env},
with setting \src{export IAB_SYS=Ubuntu-gnu-ompi} on compilation
(See \file{QuickStart.md}).


\begin{table}[htbp]
\centering
\caption{Measuring environment}\label{t:machine_env}
\small
\begin{tabularx}{.8\textwidth}{llX}
\hline
component & specification & notes \\
\hline
 CPU & Xeon E5-2630v4 @2.2GHz (10cores) x2 & HT disabled, TB enabled\\
 Memory & 256GB &\\
 Storage & SSD (SATA) &\\
 OS & Ubuntu 16.04.4 LTS &\\
 Compiler & GNU 5.4.0 & \\
\hline
\end{tabularx}
\end{table}


% \begin{table}
% \centering
% \caption{Measuring environment(Compiler, MPI)}\label{t:machine_env_2}
% \small
% \begin{tabularx}{.9\textwidth}{XXXX}
% \hline
% \src{IAB_SYS} & OS & Compiler & MPI \\
% \hline
% Ubuntu-gnu-ompi  & Ubuntu16.04 & GNU 5.4.0  & OpenMPI 1.10.2\\
% Ubuntu-pgi-ompi  & Ubuntu16.04 & PGI 17.4.0 & OpenMPI 1.10.2\\
% Ubuntu-intel-ompi& Ubuntu16.04 & Intel 2018 & OpenMPI 3.0.0\\
% \hline
% \end{tabularx}
% \end{table}
