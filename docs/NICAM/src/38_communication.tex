\section{\src{communication}}

\subsection{Description}

Kernel \src{communication} is taken from the original subroutine
\src{COMM_setup} in \NICAM.
%
This subroutine is originally defined in module \src{mod_comm}.
%
This module defines all communication stuff.
%
Subroutine \src{COMM_setup} prepares necessary information for
communication, such as connection list of regions.

As described in \autoref{s:mpi_openmpi} this kernel only needs MPI
library to compile/execute.

\subsection{Discretization and code}



\subsubsection{\src{COMM_setup}}

Main subroutine of this kernel program \src{COMM_setup} is as follows.

\begin{LstF90}
  subroutine COMM_setup
    use mod_process, only: &
       PRC_MPIstop
    use mod_adm, only: &
       RGNMNG_r2p_pl, &
       I_NPL,         &
       I_SPL
    implicit none

    namelist / COMMPARAM / &
       COMM_apply_barrier, &
       COMM_varmax,        &
       debug,              &
       testonly

    integer :: ierr
    !---------------------------------------------------------------------------

    !--- read parameters
    write(IO_FID_LOG,*)
    write(IO_FID_LOG,*) '+++ Module[comm]/Category[common share]'
    rewind(IO_FID_CONF)
    read(IO_FID_CONF,nml=COMMPARAM,iostat=ierr)
    if ( ierr < 0 ) then
       write(IO_FID_LOG,*) '*** COMMPARAM is not specified. use default.'
    elseif( ierr > 0 ) then
       write(*         ,*) 'xxx Not appropriate names in namelist COMMPARAM. STOP.'
       write(IO_FID_LOG,*) 'xxx Not appropriate names in namelist COMMPARAM. STOP.'
       call PRC_MPIstop
    endif
    write(IO_FID_LOG,nml=COMMPARAM)

    if ( RP == DP ) then
       COMM_datatype = MPI_DOUBLE_PRECISION
    elseif( RP == SP ) then
       COMM_datatype = MPI_REAL
    else
       write(*,*) 'xxx precision is not supportd'
       call PRC_MPIstop
    endif

    if (       RGNMNG_r2p_pl(I_NPL) < 0 &
         .AND. RGNMNG_r2p_pl(I_SPL) < 0 ) then
       COMM_pl = .false.
    endif

    write(IO_FID_LOG,*)
    write(IO_FID_LOG,*) '====== communication information ======'

    call COMM_list_generate

    call COMM_sortdest
    call COMM_sortdest_pl
    call COMM_sortdest_singular

    allocate( REQ_list( Recv_nmax_r2r + Send_nmax_r2r &
                      + Recv_nmax_p2r + Send_nmax_p2r &
                      + Recv_nmax_r2p + Send_nmax_r2p ) )

    if( testonly ) call COMM_debugtest

    return
  end subroutine COMM_setup
\end{LstF90}

This subroutine calls five private subroutines after reading control
namelist \src{COMMPARAM}.
%
Note that as shown below, control variable \src{testonly} is set as
true, \src{COMM_debugtest} is also called.


\subsubsection{\src{COMM_list_generate}}

Subroutine \src{COMM_list_generate} is to setup an array \src{rellist},
that contains several information about the relation between grid points
including halo.

Initial part of this subroutine is as follows.
\begin{LstF90}[name=COMM_list_generate]
  !-----------------------------------------------------------------------------
  !> Generate inner grid -> halo communication list
  subroutine COMM_list_generate
    use mod_adm, only: &
       ADM_prc_me,      &
       ADM_lall,        &
       ADM_gall,        &
       ADM_gmax,        &
       ADM_gmin,        &
       RGNMNG_r2lp,     &
       RGNMNG_l2r,      &
       RGNMNG_edge_tab, &
       RGNMNG_vert_num, &
       RGNMNG_vert_tab, &
       I_prc,           &
       I_RGNID,         &
       I_DIR,           &
       I_SW,            &
       I_NW,            &
       I_NE,            &
       I_SE,            &
       I_W,             &
       I_N,             &
       I_E,             &
       I_S
    implicit none

    integer :: ginnar

    integer :: prc, prc_rmt
    integer :: rgnid, rgnid_rmt
    integer :: i, j, i_rmt, j_rmt

    integer :: n, l, cnt
    !---------------------------------------------------------------------------

    ginnar = ADM_gmax - ADM_gmin + 1

    allocate( rellist(rellist_vindex,ADM_gall*ADM_lall) )

\end{LstF90}
%
This subroutine is private in module, there are no arguments, but using
\src{use} to include from other module.
%
Variables with prefix \src{ADM_} define problem size, and variables with
prefix \src{RGNMNG_} define tables to manage the relation between
regions and processes.
%
Variables with prefix \src{I_} are the constant index and are used to
specify the kind of quantity.
%
For example, \src{RGNMNG_r2lp(I_prc,rgnid)} means the process number
that manage given region with \src{rgnid}



The main part of this subroutine is a pretty long $l$-loop, that is
consist of two sections.
%
The first section is as follows.

\begin{LstF90}[name=COMM_list_generate,firstnumber=last]
    cnt = 0
    do l = 1, ADM_lall
       rgnid = RGNMNG_l2r(l)
       prc   = ADM_prc_me

       !---< South West >---

       ! NE -> SW halo
       if ( RGNMNG_edge_tab(I_DIR,I_SW,rgnid) == I_NE ) then
          rgnid_rmt = RGNMNG_edge_tab(I_RGNID,I_SW,rgnid)
          prc_rmt   = RGNMNG_r2lp(I_prc,rgnid_rmt)

          do n = 1, ginnar
             cnt = cnt + 1

             i     = ADM_gmin - 1 + n
             j     = ADM_gmin - 1
             i_rmt = ADM_gmin - 1 + n
             j_rmt = ADM_gmax

             rellist(I_recv_grid,cnt) = suf(i,j)
             rellist(I_recv_rgn, cnt) = rgnid
             rellist(I_recv_prc, cnt) = prc
             rellist(I_send_grid,cnt) = suf(i_rmt,j_rmt)
             rellist(I_send_rgn, cnt) = rgnid_rmt
             rellist(I_send_prc, cnt) = prc_rmt
          enddo
       endif

       ! SE -> SW halo (Southern Hemisphere, Edge of diamond)
       if ( RGNMNG_edge_tab(I_DIR,I_SW,rgnid) == I_SE ) then
          rgnid_rmt = RGNMNG_edge_tab(I_RGNID,I_SW,rgnid)
          prc_rmt   = RGNMNG_r2lp(I_prc,rgnid_rmt)

          do n = 1, ginnar
             cnt = cnt + 1

             i     = ADM_gmin - 1 + n
             j     = ADM_gmin - 1
             i_rmt = ADM_gmax
             j_rmt = ADM_gmax + 1 - n ! reverse order

             rellist(I_recv_grid,cnt) = suf(i,j)
             rellist(I_recv_rgn, cnt) = rgnid
             rellist(I_recv_prc, cnt) = prc
             rellist(I_send_grid,cnt) = suf(i_rmt,j_rmt)
             rellist(I_send_rgn, cnt) = rgnid_rmt
             rellist(I_send_prc, cnt) = prc_rmt
          enddo
       endif

       !---< North West >---

       ! SE -> NW
       if ( RGNMNG_edge_tab(I_DIR,I_NW,rgnid) == I_SE ) then
          rgnid_rmt = RGNMNG_edge_tab(I_RGNID,I_NW,rgnid)
          prc_rmt   = RGNMNG_r2lp(I_prc,rgnid_rmt)

          do n = 1, ginnar
             cnt = cnt + 1

             i     = ADM_gmin - 1
             j     = ADM_gmin - 1 + n
             i_rmt = ADM_gmax
             j_rmt = ADM_gmin - 1 + n

             rellist(I_recv_grid,cnt) = suf(i,j)
             rellist(I_recv_rgn, cnt) = rgnid
             rellist(I_recv_prc, cnt) = prc
             rellist(I_send_grid,cnt) = suf(i_rmt,j_rmt)
             rellist(I_send_rgn, cnt) = rgnid_rmt
             rellist(I_send_prc, cnt) = prc_rmt
          enddo
       endif

       ! NE -> NW  (Northern Hemisphere, Edge of diamond)
       if ( RGNMNG_edge_tab(I_DIR,I_NW,rgnid) == I_NE ) then
          rgnid_rmt = RGNMNG_edge_tab(I_RGNID,I_NW,rgnid)
          prc_rmt   = RGNMNG_r2lp(I_prc,rgnid_rmt)

          do n = 1, ginnar
             cnt = cnt + 1

             i     = ADM_gmin - 1
             j     = ADM_gmin - 1 + n
             i_rmt = ADM_gmax + 1 - n ! reverse order
             j_rmt = ADM_gmax

             rellist(I_recv_grid,cnt) = suf(i,j)
             rellist(I_recv_rgn, cnt) = rgnid
             rellist(I_recv_prc, cnt) = prc
             rellist(I_send_grid,cnt) = suf(i_rmt,j_rmt)
             rellist(I_send_rgn, cnt) = rgnid_rmt
             rellist(I_send_prc, cnt) = prc_rmt
          enddo
       endif

       !---< North East >---

       ! SW -> NE
       if ( RGNMNG_edge_tab(I_DIR,I_NE,rgnid) == I_SW ) then
          rgnid_rmt = RGNMNG_edge_tab(I_RGNID,I_NE,rgnid)
          prc_rmt   = RGNMNG_r2lp(I_prc,rgnid_rmt)

          do n = 1, ginnar
             cnt = cnt + 1

             i     = ADM_gmin - 1 + n
             j     = ADM_gmax + 1
             i_rmt = ADM_gmin - 1 + n
             j_rmt = ADM_gmin

             rellist(I_recv_grid,cnt) = suf(i,j)
             rellist(I_recv_rgn, cnt) = rgnid
             rellist(I_recv_prc, cnt) = prc
             rellist(I_send_grid,cnt) = suf(i_rmt,j_rmt)
             rellist(I_send_rgn, cnt) = rgnid_rmt
             rellist(I_send_prc, cnt) = prc_rmt
          enddo
       endif

       ! NW -> NE  (Northern Hemisphere, Edge of diamond)
       if ( RGNMNG_edge_tab(I_DIR,I_NE,rgnid) == I_NW ) then
          rgnid_rmt = RGNMNG_edge_tab(I_RGNID,I_NE,rgnid)
          prc_rmt   = RGNMNG_r2lp(I_prc,rgnid_rmt)

          do n = 1, ginnar
             cnt = cnt + 1

             i     = ADM_gmin     + n ! shift 1 grid
             j     = ADM_gmax + 1
             i_rmt = ADM_gmin
             j_rmt = ADM_gmax + 1 - n ! reverse order

             rellist(I_recv_grid,cnt) = suf(i,j)
             rellist(I_recv_rgn, cnt) = rgnid
             rellist(I_recv_prc, cnt) = prc
             rellist(I_send_grid,cnt) = suf(i_rmt,j_rmt)
             rellist(I_send_rgn, cnt) = rgnid_rmt
             rellist(I_send_prc, cnt) = prc_rmt
          enddo
       endif

       !---< South East >---

       ! NW -> SE
       if ( RGNMNG_edge_tab(I_DIR,I_SE,rgnid) == I_NW ) then
          rgnid_rmt = RGNMNG_edge_tab(I_RGNID,I_SE,rgnid)
          prc_rmt   = RGNMNG_r2lp(I_prc,rgnid_rmt)

          do n = 1, ginnar
             cnt = cnt + 1

             i     = ADM_gmax + 1
             j     = ADM_gmin - 1 + n
             i_rmt = ADM_gmin
             j_rmt = ADM_gmin - 1 + n

             rellist(I_recv_grid,cnt) = suf(i,j)
             rellist(I_recv_rgn, cnt) = rgnid
             rellist(I_recv_prc, cnt) = prc
             rellist(I_send_grid,cnt) = suf(i_rmt,j_rmt)
             rellist(I_send_rgn, cnt) = rgnid_rmt
             rellist(I_send_prc, cnt) = prc_rmt
          enddo
       endif

       ! SW -> SE  (Southern Hemisphere, Edge of diamond)
       if ( RGNMNG_edge_tab(I_DIR,I_SE,rgnid) == I_SW ) then
          rgnid_rmt = RGNMNG_edge_tab(I_RGNID,I_SE,rgnid)
          prc_rmt   = RGNMNG_r2lp(I_prc,rgnid_rmt)

          do n = 1, ginnar
             cnt = cnt + 1

             i     = ADM_gmax + 1
             j     = ADM_gmin     + n ! shift 1 grid
             i_rmt = ADM_gmax + 1 - n ! reverse order
             j_rmt = ADM_gmin

             rellist(I_recv_grid,cnt) = suf(i,j)
             rellist(I_recv_rgn, cnt) = rgnid
             rellist(I_recv_prc, cnt) = prc
             rellist(I_send_grid,cnt) = suf(i_rmt,j_rmt)
             rellist(I_send_rgn, cnt) = rgnid_rmt
             rellist(I_send_prc, cnt) = prc_rmt
          enddo
       endif

\end{LstF90}
%
This section sets \src{rellist} at four edges of each region.
%
The first dimension of \src{rellist} has the size 6, they specify
the pair of source and destination grid point.
The data is transferred from source grid to destination grid in halo communication.
%
The second dimension is for the serial number of the grid.
%
That is, \src{rellist} manages 6 kind of quantities for each inner grid,
first three are one-dimensioned grid index from $(i,j)$, region
number and process number, respectively.
The grid points identified by these three indices are the destination of data transfer.
%
The latter three are the same but of the informations of the source grid points.

Remember that the shape of the region in \NICAM is a diamond,
and setting of \src{rellist} is separated for each edge of a diamond.
%
For example, \src{( RGNMNG_edge_tab(I_DIR,I_SW,rgnid) == I_NE )}(l.49) means that
whether the south-west edge of the region with \src{rgnid} is linked with the
north-east edge of neighbouring region.
%
Therefore, this \src{IF}-clause sets \src{rellist} as that the halo
grid points of this region are linked with the corresponding inner grid points of
the region \src{rgnid_rmt} which is managed by the process
\src{prc_rmt}.

The continuing section is as follows.

\begin{LstF90}[name=COMM_list_generate,firstnumber=last]
       !---< Vertex : link to the next next region >---

       ! West Vertex
       if ( RGNMNG_vert_num(I_W,rgnid) == 4 ) then
          rgnid_rmt = RGNMNG_vert_tab(I_RGNID,I_W,rgnid,3)
          prc_rmt   = RGNMNG_r2lp(I_prc,rgnid_rmt)

          cnt = cnt + 1

          i = ADM_gmin - 1
          j = ADM_gmin - 1

          if ( RGNMNG_vert_tab(I_DIR,I_W,rgnid,3) == I_N ) then
             i_rmt = ADM_gmin
             j_rmt = ADM_gmax
          elseif( RGNMNG_vert_tab(I_DIR,I_W,rgnid,3) == I_E ) then
             i_rmt = ADM_gmax
             j_rmt = ADM_gmax
          elseif( RGNMNG_vert_tab(I_DIR,I_W,rgnid,3) == I_S ) then
             i_rmt = ADM_gmax
             j_rmt = ADM_gmin
          endif

          rellist(I_recv_grid,cnt) = suf(i,j)
          rellist(I_recv_rgn, cnt) = rgnid
          rellist(I_recv_prc, cnt) = prc
          rellist(I_send_grid,cnt) = suf(i_rmt,j_rmt)
          rellist(I_send_rgn, cnt) = rgnid_rmt
          rellist(I_send_prc, cnt) = prc_rmt
       endif

       ! North Vertex
       if ( RGNMNG_vert_num(I_N,rgnid) == 4 ) then
          rgnid_rmt = RGNMNG_vert_tab(I_RGNID,I_N,rgnid,3)
          prc_rmt   = RGNMNG_r2lp(I_prc,rgnid_rmt)

          ! known as north pole point
          if ( RGNMNG_vert_tab(I_DIR,I_N,rgnid,3) == I_W ) then
             cnt = cnt + 1

             i     = ADM_gmin
             j     = ADM_gmax + 1
             i_rmt = ADM_gmin
             j_rmt = ADM_gmin

             rellist(I_recv_grid,cnt) = suf(i,j)
             rellist(I_recv_rgn, cnt) = rgnid
             rellist(I_recv_prc, cnt) = prc
             rellist(I_send_grid,cnt) = suf(i_rmt,j_rmt)
             rellist(I_send_rgn, cnt) = rgnid_rmt
             rellist(I_send_prc, cnt) = prc_rmt
          endif

          ! unused vertex point
          cnt = cnt + 1

          i = ADM_gmin - 1
          j = ADM_gmax + 1

          if ( RGNMNG_vert_tab(I_DIR,I_N,rgnid,3) == I_W ) then
             i_rmt = ADM_gmin
             j_rmt = ADM_gmin + 1
          elseif( RGNMNG_vert_tab(I_DIR,I_N,rgnid,3) == I_N ) then
             i_rmt = ADM_gmin
             j_rmt = ADM_gmax
          elseif( RGNMNG_vert_tab(I_DIR,I_N,rgnid,3) == I_E ) then
             i_rmt = ADM_gmax
             j_rmt = ADM_gmax
          elseif( RGNMNG_vert_tab(I_DIR,I_N,rgnid,3) == I_S ) then
             i_rmt = ADM_gmax
             j_rmt = ADM_gmin
          endif

          rellist(I_recv_grid,cnt) = suf(i,j)
          rellist(I_recv_rgn, cnt) = rgnid
          rellist(I_recv_prc, cnt) = prc
          rellist(I_send_grid,cnt) = suf(i_rmt,j_rmt)
          rellist(I_send_rgn, cnt) = rgnid_rmt
          rellist(I_send_prc, cnt) = prc_rmt
       endif

       ! East Vertex
       if ( RGNMNG_vert_num(I_E,rgnid) == 4 ) then
          if ( RGNMNG_vert_tab(I_DIR,I_E,rgnid,3) == I_W ) then
             rgnid_rmt = RGNMNG_vert_tab(I_RGNID,I_E,rgnid,3)
             prc_rmt   = RGNMNG_r2lp(I_prc,rgnid_rmt)

             cnt = cnt + 1

             i     = ADM_gmax + 1
             j     = ADM_gmax + 1
             i_rmt = ADM_gmin
             j_rmt = ADM_gmin

             rellist(I_recv_grid,cnt) = suf(i,j)
             rellist(I_recv_rgn, cnt) = rgnid
             rellist(I_recv_prc, cnt) = prc
             rellist(I_send_grid,cnt) = suf(i_rmt,j_rmt)
             rellist(I_send_rgn, cnt) = rgnid_rmt
             rellist(I_send_prc, cnt) = prc_rmt
          endif
       endif

       ! South Vertex
       if ( RGNMNG_vert_num(I_S,rgnid) == 4 ) then
          rgnid_rmt = RGNMNG_vert_tab(I_RGNID,I_S,rgnid,3)
          prc_rmt   = RGNMNG_r2lp(I_prc,rgnid_rmt)

          ! known as south pole point
          if ( RGNMNG_vert_tab(I_DIR,I_S,rgnid,3) == I_W ) then
             cnt = cnt + 1

             i     = ADM_gmax + 1
             j     = ADM_gmin
             i_rmt = ADM_gmin
             j_rmt = ADM_gmin

             rellist(I_recv_grid,cnt) = suf(i,j)
             rellist(I_recv_rgn, cnt) = rgnid
             rellist(I_recv_prc, cnt) = prc
             rellist(I_send_grid,cnt) = suf(i_rmt,j_rmt)
             rellist(I_send_rgn, cnt) = rgnid_rmt
             rellist(I_send_prc, cnt) = prc_rmt
          endif

          ! unused vertex point
          cnt = cnt + 1

          i = ADM_gmax + 1
          j = ADM_gmin - 1

          if ( RGNMNG_vert_tab(I_DIR,I_S,rgnid,3) == I_W ) then
             i_rmt = ADM_gmin + 1
             j_rmt = ADM_gmin
          elseif( RGNMNG_vert_tab(I_DIR,I_S,rgnid,3) == I_N ) then
             i_rmt = ADM_gmin
             j_rmt = ADM_gmax
          elseif( RGNMNG_vert_tab(I_DIR,I_S,rgnid,3) == I_E ) then
             i_rmt = ADM_gmax
             j_rmt = ADM_gmax
          elseif( RGNMNG_vert_tab(I_DIR,I_S,rgnid,3) == I_S ) then
             i_rmt = ADM_gmax
             j_rmt = ADM_gmin
          endif

          rellist(I_recv_grid,cnt) = suf(i,j)
          rellist(I_recv_rgn, cnt) = rgnid
          rellist(I_recv_prc, cnt) = prc
          rellist(I_send_grid,cnt) = suf(i_rmt,j_rmt)
          rellist(I_send_rgn, cnt) = rgnid_rmt
          rellist(I_send_prc, cnt) = prc_rmt
       endif
    enddo ! loop l

    rellist_nmax = cnt

    write(IO_FID_LOG,*)
    write(IO_FID_LOG,*) "*** rellist_nmax:", rellist_nmax

    if ( debug ) then
       write(IO_FID_LOG,*) "--- Relation Table"
       write(IO_FID_LOG,'(7(A10))') 'Count', '|recv_grid', '| recv_rgn', '| recv_prc', &
                                                       '|send_grid', '| send_rgn', '| send_prc'
       do cnt = 1, rellist_nmax
          write(IO_FID_LOG,'(7(I10))') cnt, rellist(:,cnt)
       enddo
    endif

    return
  end subroutine COMM_list_generate
\end{LstF90}
%
Similar to the previous secion, this section sets \src{rellist} additional source-destination relationships of each region.
Around the vertices, the data can be transferred from next-next region.
%
In the \src{IF}-clause at l.233, \src{( RGNMNG_vert_num(I_W,rgnid) == 4 )} means whether the number of
regions around the western vertex of the region with \src{rgnid} is four, i.e.
this vertex is not a singular point, that has only three region links.
%
And \src{( RGNMNG_vert_tab(I_DIR,I_W,rgnid,3) == I_N )}(l.242) means whether
the northern vertex of the next-next region is connected to the western vertex of current region.
The regions around the vertex are searched clockwise.
%
Other two \src{IF}-clauses(l.245 and l.248) are the same meaning
but for the east or west vertex of the next-next region.

At the northern and southern vertices, there are the cases applying special treatment.
If the region does not contain north pole at the halo,
and the north-western edge is connected to the north-eastern edge of neighboring region,
We should take one grid from the next-next region into account.
%
Additionally, in northern and southern vertex section, northern-most and southern-most vertex in each region
is not used, respectively (See \autoref{f:grid_structure_of_a_region} in \autoref{s:horiz_data_structure}).
These points are filled by the value of the grid in the self region.

After the $l$-loop, contents of \src{rellist} is written to the log
(standard output).
%
See \autoref{s:comm_input_result} for example.

Also note that even if one region is managed by a single process, all
process has the same \src{rellist} that contains the information of
\emph{ALL} regions.


\subsubsection{\src{COMM_sortdest}}

Subroutine \src{COMM_sortdest} is to sort the region to region
connection info.

The first part is as follows.

\begin{LstF90}[name=COMM_sortdest]
!-----------------------------------------------------------------------------
!> Sort data destination for region <-> region
subroutine COMM_sortdest
  use mod_process, only: &
     PRC_LOCAL_COMM_WORLD, &
     PRC_nprocs
  use mod_adm, only: &
     ADM_prc_me,     &
     ADM_kall,       &
     ADM_gall_1d,    &
     RGNMNG_lp2r,    &
     RGNMNG_r2lp,    &
     I_l
  implicit none

  integer :: sendbuf1(1)
  integer :: recvbuf1(1)

  integer, allocatable :: sendbuf_info(:)
  integer, allocatable :: recvbuf_info(:)
  integer, allocatable :: sendbuf_list(:,:,:)
  integer, allocatable :: recvbuf_list(:,:,:)
  integer, allocatable :: REQ_list_r2r(:)

  integer :: Recv_nglobal_r2r
  integer :: Send_size_nglobal

  integer :: cnt, irank, ipos
  integer :: totalsize, rank, tag

  integer :: i_from, j_from, r_from, g_from, l_from, p_from
  integer :: i_to, j_to, r_to, g_to, l_to, p_to

  integer :: ierr
  integer :: n, p
  !---------------------------------------------------------------------------

  allocate( Copy_info_r2r(info_vindex) )
  allocate( Recv_info_r2r(info_vindex,Recv_nlim) )
  allocate( Send_info_r2r(info_vindex,Send_nlim) )
  Copy_info_r2r(:)        = -1
  Recv_info_r2r(:,:)      = -1
  Send_info_r2r(:,:)      = -1
  Copy_info_r2r(I_size)   = 0
  Recv_info_r2r(I_size,:) = 0
  Send_info_r2r(I_size,:) = 0

  allocate( Copy_list_r2r(list_vindex,rellist_nmax) )
  allocate( Recv_list_r2r(list_vindex,rellist_nmax,Recv_nlim) )
  allocate( Send_list_r2r(list_vindex,rellist_nmax,Send_nlim) )
  Copy_list_r2r(:,:)   = -1
  Recv_list_r2r(:,:,:) = -1
  Send_list_r2r(:,:,:) = -1

\end{LstF90}
%
This subroutine is also private in this module, there are no arguments.
%
Variables with the prefix \src{sendbuf} and \src{recvbuf} are temporal
containers used for MPI communication.
%
\src{Send_info_r2r} and \src{Recv_info_r2r} specify
the number of transferring grid points between two precesses, for the sending side and receiving side, respectively.
\src{Copy_info_r2r} is the number of grid points to copy within the same process.
%
\src{Copy_info_r2r} is one-dimensional and the size is \src{info_vindex}
that is $3$, and the index takes one of \src{I_size}, \src{I_prc_from} or
\src{I_prc_to}.
\src{Copy_info_r2r(I_size)} is the number of halo-grid points that exchange with
the grid points in other region by memory-copy, and \src{Copy_info_r2r(I_prc_from)} and
\src{Copy_info_r2r(i_prc_to)} are the process number that the process
this region exchange from/to, of course this value is the same with
\src{ADM_prc_me}, the process this region is managed.

On the other hand, \src{Recv_info_r2r} and \src{Send_info_r2r} are
two-dimensional, the first dimension of them are the same with
\src{Copy_info_r2r}.
%
The size of the second dimension of these are \src{Recv_nlim} and
\src{Send_nlim}, respectively, they are both $10$, and this dimension
specifies that which process this process receive from/send to.

\src{Copy_list_r2r} is two-dimensional, the size of these dimension are
\src{list_vindex}, which is $4$, and \src{rellist_nmax}, which is the
number of grid points \src{rellist} contains.
%
First dimension takes one of \src{I_grid_from}, \src{I_l_from}
\src{I_grid_to} or \src{I_l_to}, they specify
the grid point and the (local) region number the value comes from,
the grid point and the (local) region number the value goes to, respectively.
%
And second dimension specifies the grid point in this process whose value
is exchanged by the memory-copy.

\src{Recv_list_r2r} and \src{Send_list_r2r} are similar, but have
another dimension, whose size are \src{Recv_nlim} and \src{Send_nlim},
respectively.
%
This last dimension has the same meaning with the last dimension of
\src{Recv_info_r2r} and \src{Send_info_r2r}.


These arrays are set in the next section, as follows.
\begin{LstF90}[name=COMM_sortdest,firstnumber=last]
  ! sorting list according to destination
  do cnt = 1, rellist_nmax

     if ( rellist(I_recv_prc,cnt) == rellist(I_send_prc,cnt) ) then ! no communication
        Copy_info_r2r(I_size) = Copy_info_r2r(I_size) + 1
        ipos                  = Copy_info_r2r(I_size)

        Copy_list_r2r(I_grid_from,ipos) = rellist(I_send_grid,cnt)
        Copy_list_r2r(I_l_from   ,ipos) = RGNMNG_r2lp(I_l,rellist(I_send_rgn,cnt))
        Copy_list_r2r(I_grid_to  ,ipos) = rellist(I_recv_grid,cnt)
        Copy_list_r2r(I_l_to     ,ipos) = RGNMNG_r2lp(I_l,rellist(I_recv_rgn,cnt))
     else ! node-to-node communication
        !--- search existing rank id (identify by prc_from)
        irank = -1
        do n = 1, Recv_nmax_r2r
           if ( Recv_info_r2r(I_prc_from,n) == rellist(I_send_prc,cnt) ) then
              irank = n
              exit
           endif
        enddo

        if ( irank < 0 ) then ! register new rank id
           Recv_nmax_r2r = Recv_nmax_r2r + 1
           irank         = Recv_nmax_r2r

           Recv_info_r2r(I_prc_from,irank) = rellist(I_send_prc,cnt)
           Recv_info_r2r(I_prc_to  ,irank) = rellist(I_recv_prc,cnt)
        endif

        Recv_info_r2r(I_size,irank) = Recv_info_r2r(I_size,irank) + 1
        ipos                        = Recv_info_r2r(I_size,irank)

        Recv_list_r2r(I_grid_from,ipos,irank) = rellist(I_send_grid,cnt)
        Recv_list_r2r(I_l_from   ,ipos,irank) = RGNMNG_r2lp(I_l,rellist(I_send_rgn,cnt))
        Recv_list_r2r(I_grid_to  ,ipos,irank) = rellist(I_recv_grid,cnt)
        Recv_list_r2r(I_l_to     ,ipos,irank) = RGNMNG_r2lp(I_l,rellist(I_recv_rgn,cnt))
     endif
  enddo

  if ( Copy_info_r2r(I_size) > 0 ) then
     Copy_nmax_r2r = 1
     Copy_info_r2r(I_prc_from) = ADM_prc_me
     Copy_info_r2r(I_prc_to  ) = ADM_prc_me
  endif



\end{LstF90}


In the next section there are some MPI communication, as follows.

\begin{LstF90}[name=COMM_sortdest,firstnumber=last]
  ! get maximum number of rank to communication
  sendbuf1(1) = Recv_nmax_r2r

  call MPI_Allreduce( sendbuf1(1),          & ! [IN]
                      recvbuf1(1),          & ! [OUT]
                      1,                    & ! [IN]
                      MPI_INTEGER,          & ! [IN]
                      MPI_MAX,              & ! [IN]
                      PRC_LOCAL_COMM_WORLD, & ! [IN]
                      ierr                  ) ! [OUT]

  Recv_nglobal_r2r = recvbuf1(1)

  ! Distribute receive request from each rank to all
  allocate( sendbuf_info(info_vindex*Recv_nglobal_r2r) )
  allocate( recvbuf_info(info_vindex*Recv_nglobal_r2r*PRC_nprocs) )
  sendbuf_info(:) = -1

  do irank = 1, Recv_nmax_r2r
     n = (irank-1) * info_vindex

     sendbuf_info(n+I_size    ) = Recv_info_r2r(I_size    ,irank)
     sendbuf_info(n+I_prc_from) = Recv_info_r2r(I_prc_from,irank)
     sendbuf_info(n+I_prc_to  ) = Recv_info_r2r(I_prc_to  ,irank)
  enddo

  totalsize = info_vindex * Recv_nglobal_r2r

  call MPI_Allgather( sendbuf_info(1),      & ! [IN]
                      totalsize,            & ! [IN]
                      MPI_INTEGER,          & ! [IN]
                      recvbuf_info(1),      & ! [OUT]
                      totalsize,            & ! [IN]
                      MPI_INTEGER,          & ! [IN]
                      PRC_LOCAL_COMM_WORLD, & ! [IN]
                      ierr                  ) ! [OUT]

  Send_size_nglobal = 0

  ! Accept receive request to my rank
  do p  = 1, Recv_nglobal_r2r*PRC_nprocs
     n = (p-1) * info_vindex

     if ( recvbuf_info(n+I_prc_from) == ADM_prc_me ) then
        Send_nmax_r2r = Send_nmax_r2r + 1
        irank         = Send_nmax_r2r

        Send_info_r2r(I_size    ,irank) = recvbuf_info(n+I_size    )
        Send_info_r2r(I_prc_from,irank) = recvbuf_info(n+I_prc_from)
        Send_info_r2r(I_prc_to  ,irank) = recvbuf_info(n+I_prc_to  )
     endif

     Send_size_nglobal = max( Send_size_nglobal, recvbuf_info(n+I_size) )
  enddo

  write(IO_FID_LOG,*)
  write(IO_FID_LOG,*) '*** Recv_nmax_r2r(global) = ', Recv_nglobal_r2r
  write(IO_FID_LOG,*) '*** Recv_nmax_r2r(local)  = ', Recv_nmax_r2r
  write(IO_FID_LOG,*) '*** Send_nmax_r2r(local)  = ', Send_nmax_r2r
  write(IO_FID_LOG,*) '*** Send_size_r2r(global) = ', Send_size_nglobal
  write(IO_FID_LOG,*)
  write(IO_FID_LOG,'(A)')             "|---------------------------------------"
  write(IO_FID_LOG,'(A)')             "|               size  prc_from    prc_to"
  write(IO_FID_LOG,'(A10,3(I10))')    "| Copy_r2r", Copy_info_r2r(:)
  do irank = 1, Recv_nmax_r2r
     write(IO_FID_LOG,'(A10,3(I10))') "| Recv_r2r", Recv_info_r2r(:,irank)
  enddo
  do irank = 1, Send_nmax_r2r
     write(IO_FID_LOG,'(A10,3(I10))') "| Send_r2r", Send_info_r2r(:,irank)
  enddo

\end{LstF90}
%
In this section, shown as several write statements,
\src{Recv_nglobal_r2r}, \src{Recv_nmax_r2r}, \src{Send_nmax_r2r} and
\src{Send_size_nglobal} are calculated.
%
\src{Recv_nglobal_r2r} is the maximum of \src{Recv_nmax_r2r} in all
processes, number of processes to communicate.
%
This value is calculated by \src{MPI_Allreduce} with \src{MPI_MAX}
operation.
%
Then contents of \src{Recv_info_r2r} are \src{MPI_Allgather}ed and the
contents of \src{Recv_info_r2r} which related to the current process are
additionally stored to the \src{Send_info_r2r}.


The next section is as follows.

\begin{LstF90}[name=COMM_sortdest,firstnumber=last]
  ! Communicate detailed information in each pair
  allocate( REQ_list_r2r(Recv_nmax_r2r+Send_nmax_r2r)  )

  allocate( sendbuf_list(list_vindex,Send_size_nglobal,Recv_nmax_r2r) )
  allocate( recvbuf_list(list_vindex,Send_size_nglobal,Send_nmax_r2r) )
  sendbuf_list(:,:,:) = -1

  REQ_count = 0

  do irank = 1, Send_nmax_r2r
     REQ_count = REQ_count + 1
     totalsize = Send_info_r2r(I_size,irank) * list_vindex
     rank      = Send_info_r2r(I_prc_to  ,irank) - 1 ! rank = prc - 1
     tag       = Send_info_r2r(I_prc_from,irank) - 1 ! rank = prc - 1

     call MPI_IRECV( recvbuf_list(1,1,irank), & ! [OUT]
                     totalsize,               & ! [IN]
                     MPI_INTEGER,             & ! [IN]
                     rank,                    & ! [IN]
                     tag,                     & ! [IN]
                     PRC_LOCAL_COMM_WORLD,    & ! [IN]
                     REQ_list_r2r(REQ_count), & ! [OUT]
                     ierr                     ) ! [OUT]
  enddo

  do irank = 1, Recv_nmax_r2r
     do ipos = 1, Recv_info_r2r(I_size,irank)
        sendbuf_list(:,ipos,irank) = Recv_list_r2r(:,ipos,irank)
     enddo

     REQ_count = REQ_count + 1
     totalsize = Recv_info_r2r(I_size,irank) * list_vindex
     rank      = Recv_info_r2r(I_prc_from,irank) - 1 ! rank = prc - 1
     tag       = Recv_info_r2r(I_prc_from,irank) - 1 ! rank = prc - 1

     call MPI_ISEND( sendbuf_list(1,1,irank), & ! [IN]
                     totalsize,               & ! [IN]
                     MPI_INTEGER,             & ! [IN]
                     rank,                    & ! [IN]
                     tag,                     & ! [IN]
                     PRC_LOCAL_COMM_WORLD,    & ! [IN]
                     REQ_list_r2r(REQ_count), & ! [OUT]
                     ierr                     ) ! [OUT]
  enddo

  !--- wait packets
  call MPI_WAITALL( Recv_nmax_r2r+Send_nmax_r2r, & ! [IN]
                    REQ_list_r2r(:),             & ! [IN]
                    MPI_STATUSES_IGNORE,         & ! [OUT]
                    ierr                         ) ! [OUT]

  do irank = 1, Send_nmax_r2r
     do ipos = 1, Send_info_r2r(I_size,irank)
        Send_list_r2r(:,ipos,irank) = recvbuf_list(:,ipos,irank)
     enddo
  enddo

\end{LstF90}

In this section each process send the contents of \src{Recv_info_r2r} to
the corresponding process, then receive and store them to the
corresponding location of \src{Send_info_r2r} and \src{Send_list_r2r}.
%
Note that this procedure is a kind of all to all communication,
implemented asynchronous communication, \src{MPI_IRECV}, \src{MPI_ISEND}
and \src{MPI_WAITALL}.

Final section of this subroutine is as follows.

\begin{LstF90}[name=COMM_sortdest,firstnumber=last]
  if ( debug ) then
     write(IO_FID_LOG,*)
     write(IO_FID_LOG,*) "--- Copy_list_r2r"
     write(IO_FID_LOG,*)
     write(IO_FID_LOG,'(13(A6))') "number", &
                                             "|ifrom","|jfrom","|rfrom","|gfrom","|lfrom","|pfrom", &
                                             "|  ito","|  jto","|  rto","|  gto","|  lto","|  pto"
     do ipos = 1, Copy_info_r2r(I_size)
        g_from = Copy_list_r2r(I_grid_from,ipos)
        l_from = Copy_list_r2r(I_l_from,ipos)
        p_from = Copy_info_r2r(I_prc_from)
        i_from = mod(g_from-1,ADM_gall_1d) + 1
        j_from = (g_from-i_from) / ADM_gall_1d + 1
        r_from = RGNMNG_lp2r(l_from,p_from)
        g_to   = Copy_list_r2r(I_grid_to,ipos)
        l_to   = Copy_list_r2r(I_l_to,ipos)
        p_to   = Copy_info_r2r(I_prc_to)
        i_to   = mod(g_to-1,ADM_gall_1d) + 1
        j_to   = (g_to-i_to) / ADM_gall_1d + 1
        r_to   = RGNMNG_lp2r(l_to,p_to)

        write(IO_FID_LOG,'(13(I6))') ipos, i_from, j_from, r_from, g_from, l_from, p_from, &
                                           i_to  , j_to  , r_to  , g_to  , l_to  , p_to
     enddo

     write(IO_FID_LOG,*)
     write(IO_FID_LOG,*) "--- Recv_list_r2r"
     do irank = 1, Recv_nmax_r2r
        write(IO_FID_LOG,'(13(A6))') "number", &
                                                "|ifrom","|jfrom","|rfrom","|gfrom","|lfrom","|pfrom", &
                                                "|  ito","|  jto","|  rto","|  gto","|  lto","|  pto"
        do ipos = 1, Recv_info_r2r(I_size,irank)
           g_from = Recv_list_r2r(I_grid_from,ipos,irank)
           l_from = Recv_list_r2r(I_l_from,ipos,irank)
           p_from = Recv_info_r2r(I_prc_from,irank)
           i_from = mod(g_from-1,ADM_gall_1d) + 1
           j_from = (g_from-i_from) / ADM_gall_1d + 1
           r_from = RGNMNG_lp2r(l_from,p_from)
           g_to   = Recv_list_r2r(I_grid_to,ipos,irank)
           l_to   = Recv_list_r2r(I_l_to,ipos,irank)
           p_to   = Recv_info_r2r(I_prc_to,irank)
           i_to   = mod(g_to-1,ADM_gall_1d) + 1
           j_to   = (g_to-i_to) / ADM_gall_1d + 1
           r_to   = RGNMNG_lp2r(l_to,p_to)

           write(IO_FID_LOG,'(13(I6))') ipos, i_from, j_from, r_from, g_from, l_from, p_from, &
                                              i_to  , j_to  , r_to  , g_to  , l_to  , p_to
        enddo
     enddo

     write(IO_FID_LOG,*)
     write(IO_FID_LOG,*) "--- Send_list_r2r"
     do irank = 1, Send_nmax_r2r
        write(IO_FID_LOG,'(13(A6))') "number", &
                                                "|ifrom","|jfrom","|rfrom","|gfrom","|lfrom","|pfrom", &
                                                "|  ito","|  jto","|  rto","|  gto","|  lto","|  pto"
        do ipos = 1, Send_info_r2r(I_size,irank)
           g_from = Send_list_r2r(I_grid_from,ipos,irank)
           l_from = Send_list_r2r(I_l_from,ipos,irank)
           p_from = Send_info_r2r(I_prc_from,irank)
           i_from = mod(g_from-1,ADM_gall_1d) + 1
           j_from = (g_from-i_from) / ADM_gall_1d + 1
           r_from = RGNMNG_lp2r(l_from,p_from)
           g_to   = Send_list_r2r(I_grid_to,ipos,irank)
           l_to   = Send_list_r2r(I_l_to,ipos,irank)
           p_to   = Send_info_r2r(I_prc_to,irank)
           i_to   = mod(g_to-1,ADM_gall_1d) + 1
           j_to   = (g_to-i_to) / ADM_gall_1d + 1
           r_to   = RGNMNG_lp2r(l_to,p_to)

           write(IO_FID_LOG,'(13(I6))') ipos, i_from, j_from, r_from, g_from, l_from, p_from, &
                                              i_to  , j_to  , r_to  , g_to  , l_to  , p_to
        enddo
     enddo
  endif

  allocate( sendbuf_r2r(Send_size_nglobal*ADM_kall*COMM_varmax,Send_nmax_r2r) )
  allocate( recvbuf_r2r(Send_size_nglobal*ADM_kall*COMM_varmax,Recv_nmax_r2r) )

  return
end subroutine COMM_sortdest
\end{LstF90}
%
This section writes the contents of \src{Copy_list_r2r},
\src{Recv_list_r2r}, \src{Send_list_r2r}, in the form of table,
respectively.
%
Note that \src{debug} is true in this kernel program.
%
See \autoref{s:comm_input_result} for the example of this output.

Regarding to subroutine \src{COMM_sortdest_pl} and
\src{COMM_sortdest_singular}, they do similar procedure with
\src{COMM_sortdest},
omit here.



\subsubsection{\src{COMM_debugtest}}

\src{COMM_debugtest} is to test actual halo-exchange communication
routine, \src{COMM_data_transfer} and \src{COMM_var}.

The first section is as follows.

\begin{LstF90}[name=COMM_debugtest]
subroutine COMM_debugtest
  use mod_process, only: &
     PRC_MPIfinish
  use mod_adm, only: &
     ADM_prc_me,      &
     ADM_gall,        &
     ADM_gall_pl,     &
     ADM_kall,        &
     ADM_lall,        &
     ADM_lall_pl,     &
     ADM_lall_pl,     &
     ADM_gall_1d,     &
     ADM_gmin,        &
     ADM_gmax,        &
     ADM_kmin,        &
     ADM_kmax,        &
     ADM_vlink,       &
     ADM_have_sgp,    &
     RGNMNG_l2r,      &
     RGNMNG_vert_num, &
     I_N,             &
     I_S
  implicit none

  real(RP) :: var   (ADM_gall   ,ADM_kall,ADM_lall   ,4)
  real(RP) :: var_pl(ADM_gall_pl,ADM_kall,ADM_lall_pl,4)

  integer  :: i, j, k, l, ij, rgnid, prc
  !---------------------------------------------------------------------------

\end{LstF90}
%
As the other subroutines in this section, this is also private in this
module, and there are no arguments.
%
\src{var} and \src{var_pl} are temporal array used in communication.

Main part are separated by two section. The fist section is as follows.

\begin{LstF90}[name=COMM_debugtest,firstnumber=last]
  write(IO_FID_LOG,*)
  write(IO_FID_LOG,*) '+++ TEST start'

  var   (:,:,:,:) = -999.D0
  var_pl(:,:,:,:) = -999.D0

  do l = 1, ADM_lall
     rgnid = RGNMNG_l2r(l)
     prc   = ADM_prc_me

     do k = ADM_kmin, ADM_kmax
        do j = ADM_gmin, ADM_gmax
        do i = ADM_gmin, ADM_gmax
           ij = ADM_gall_1d * (j-1) + i

           var(ij,k,l,1) = real(prc,  kind=RP)
           var(ij,k,l,2) = real(rgnid,kind=RP)
           var(ij,k,l,3) = real(i,    kind=RP)
           var(ij,k,l,4) = real(j,    kind=RP)
        enddo
        enddo
     enddo

     if ( ADM_have_sgp(l) ) then
        do k = ADM_kmin, ADM_kmax
           var(1,k,l,:) = -1.D0
        enddo
     endif
  enddo

  do l = 1, ADM_lall_pl
     rgnid = l
     prc   = ADM_prc_me

     do k  = ADM_kmin, ADM_kmax
     do ij = 1, ADM_gall_pl
        var_pl(ij,k,l,1) = real(-prc,  kind=RP)
        var_pl(ij,k,l,2) = real(-rgnid,kind=RP)
        var_pl(ij,k,l,3) = real(-ij,   kind=RP)
        var_pl(ij,k,l,4) = real(-ij,   kind=RP)
     enddo
     enddo
  enddo

  write(IO_FID_LOG,*) "##### (prc,rgnid) #####"
  do l  = 1, ADM_lall
  do k = ADM_kmin, ADM_kmin
     write(IO_FID_LOG,*)
     write(IO_FID_LOG,'(A9)',advance='no') "        |"
     do i = 1, ADM_gall_1d
        write(IO_FID_LOG,'(I9)',advance='no') i
     enddo
     write(IO_FID_LOG,*)

     do j = ADM_gall_1d, 1, -1
        write(IO_FID_LOG,'(I8,A1)',advance='no') j, "|"
        do i = 1, ADM_gall_1d
           ij = ADM_gall_1d * (j-1) + i
           write(IO_FID_LOG,'(A1,I3,A1,I3,A1)',advance='no') &
                      '(',int(var(ij,k,l,1)),',',int(var(ij,k,l,2)),')'
        enddo
        write(IO_FID_LOG,*)
     enddo
  enddo
  enddo

  do l  = 1, ADM_lall_pl
  do k = ADM_kmin, ADM_kmin
     write(IO_FID_LOG,*)
     write(IO_FID_LOG,'(A9)',advance='no') "        |"
     do i = 1, ADM_gall_pl
        write(IO_FID_LOG,'(I9)',advance='no') i
     enddo
     write(IO_FID_LOG,*)

     write(IO_FID_LOG,'(I8,A1)',advance='no') j, "|"
     do ij = 1, ADM_gall_pl
        write(IO_FID_LOG,'(A1,I3,A1,I3,A1)',advance='no') &
                   '(',int(var_pl(ij,k,l,1)),',',int(var_pl(ij,k,l,2)),')'
     enddo
     write(IO_FID_LOG,*)
  enddo
  enddo

  write(IO_FID_LOG,*) "##### (i,j) #####"
  do l  = 1, ADM_lall
  do k = ADM_kmin, ADM_kmin
     write(IO_FID_LOG,*)
     write(IO_FID_LOG,'(A9)',advance='no') "        |"
     do i = 1, ADM_gall_1d
        write(IO_FID_LOG,'(I9)',advance='no') i
     enddo
     write(IO_FID_LOG,*)

     do j = ADM_gall_1d, 1, -1
        write(IO_FID_LOG,'(I8,A1)',advance='no') j, "|"
        do i = 1, ADM_gall_1d
           ij = ADM_gall_1d * (j-1) + i
           write(IO_FID_LOG,'(A1,I3,A1,I3,A1)',advance='no') &
                      '(',int(var(ij,k,l,3)),',',int(var(ij,k,l,4)),')'
        enddo
        write(IO_FID_LOG,*)
     enddo
  enddo
  enddo

  do l  = 1, ADM_lall_pl
  do k = ADM_kmin, ADM_kmin
     write(IO_FID_LOG,*)
     write(IO_FID_LOG,'(A9)',advance='no') "        |"
     do i = 1, ADM_gall_pl
        write(IO_FID_LOG,'(I9)',advance='no') i
     enddo
     write(IO_FID_LOG,*)

     write(IO_FID_LOG,'(I8,A1)',advance='no') j, "|"
     do ij = 1, ADM_gall_pl
        write(IO_FID_LOG,'(A1,I3,A1,I3,A1)',advance='no') &
                   '(',int(var_pl(ij,k,l,3)),',',int(var_pl(ij,k,l,4)),')'
     enddo
     write(IO_FID_LOG,*)
  enddo
  enddo



  write(IO_FID_LOG,*)
  write(IO_FID_LOG,*) '+++ Communication start'

  call COMM_data_transfer( var(:,:,:,:), var_pl(:,:,:,:) )

  write(IO_FID_LOG,*) "##### (prc,rgnid) #####"
  do l  = 1, ADM_lall
  do k = ADM_kmin, ADM_kmin
     write(IO_FID_LOG,*)
     write(IO_FID_LOG,'(A9)',advance='no') "        |"
     do i = 1, ADM_gall_1d
        write(IO_FID_LOG,'(I9)',advance='no') i
     enddo
     write(IO_FID_LOG,*)

     do j = ADM_gall_1d, 1, -1
        write(IO_FID_LOG,'(I8,A1)',advance='no') j, "|"
        do i = 1, ADM_gall_1d
           ij = ADM_gall_1d * (j-1) + i
           write(IO_FID_LOG,'(A1,I3,A1,I3,A1)',advance='no') &
                      '(',int(var(ij,k,l,1)),',',int(var(ij,k,l,2)),')'
        enddo
        write(IO_FID_LOG,*)
     enddo
  enddo
  enddo

  do l  = 1, ADM_lall_pl
  do k = ADM_kmin, ADM_kmin
     write(IO_FID_LOG,*)
     write(IO_FID_LOG,'(A9)',advance='no') "        |"
     do i = 1, ADM_gall_pl
        write(IO_FID_LOG,'(I9)',advance='no') i
     enddo
     write(IO_FID_LOG,*)

     write(IO_FID_LOG,'(I8,A1)',advance='no') j, "|"
     do ij = 1, ADM_gall_pl
        write(IO_FID_LOG,'(A1,I3,A1,I3,A1)',advance='no') &
                   '(',int(var_pl(ij,k,l,1)),',',int(var_pl(ij,k,l,2)),')'
     enddo
     write(IO_FID_LOG,*)
  enddo
  enddo

  write(IO_FID_LOG,*) "##### (i,j) #####"
  do l  = 1, ADM_lall
  do k = ADM_kmin, ADM_kmin
     write(IO_FID_LOG,*)
     write(IO_FID_LOG,'(A9)',advance='no') "        |"
     do i = 1, ADM_gall_1d
        write(IO_FID_LOG,'(I9)',advance='no') i
     enddo
     write(IO_FID_LOG,*)

     do j = ADM_gall_1d, 1, -1
        write(IO_FID_LOG,'(I8,A1)',advance='no') j, "|"
        do i = 1, ADM_gall_1d
           ij = ADM_gall_1d * (j-1) + i
           write(IO_FID_LOG,'(A1,I3,A1,I3,A1)',advance='no') &
                      '(',int(var(ij,k,l,3)),',',int(var(ij,k,l,4)),')'
        enddo
        write(IO_FID_LOG,*)
     enddo
  enddo
  enddo

  do l  = 1, ADM_lall_pl
  do k = ADM_kmin, ADM_kmin
     write(IO_FID_LOG,*)
     write(IO_FID_LOG,'(A9)',advance='no') "        |"
     do i = 1, ADM_gall_pl
        write(IO_FID_LOG,'(I9)',advance='no') i
     enddo
     write(IO_FID_LOG,*)

     write(IO_FID_LOG,'(I8,A1)',advance='no') j, "|"
     do ij = 1, ADM_gall_pl
        write(IO_FID_LOG,'(A1,I3,A1,I3,A1)',advance='no') &
                   '(',int(var_pl(ij,k,l,3)),',',int(var_pl(ij,k,l,4)),')'
     enddo
     write(IO_FID_LOG,*)
  enddo
  enddo



\end{LstF90}
%
This section does the test of \src{COMM_data_transfer}.
%
As test data, process number \src{prc}, region number \src{rgnid},
and the grid index \src{i}, \src{j} are set to \src{var} and
\src{var_pl}.
%
The values of these arrays of before and after the calling
\src{COMM_data_transfer} are written to the log (standard out).

The latter section is as follows.

\begin{LstF90}[name=COMM_debugtest,firstnumber=last]
  do l = 1, ADM_lall
     rgnid = RGNMNG_l2r(l)
     prc   = ADM_prc_me

     if ( RGNMNG_vert_num(I_N,rgnid) == ADM_vlink ) then
        do k = ADM_kmin, ADM_kmax
           j  = ADM_gmax+1
           i  = ADM_gmin
           ij = ADM_gall_1d * (j-1) + i

           var(ij,k,l,1) = real(prc,  kind=RP)
           var(ij,k,l,2) = real(rgnid,kind=RP)
           var(ij,k,l,3) = real(i,    kind=RP)
           var(ij,k,l,4) = real(j,    kind=RP)
        enddo
     endif

     if ( RGNMNG_vert_num(I_S,rgnid) == ADM_vlink ) then
        do k = ADM_kmin, ADM_kmax
           j  = ADM_gmin
           i  = ADM_gmax+1
           ij = ADM_gall_1d * (j-1) + i

           var(ij,k,l,1) = real(prc,  kind=RP)
           var(ij,k,l,2) = real(rgnid,kind=RP)
           var(ij,k,l,3) = real(i,    kind=RP)
           var(ij,k,l,4) = real(j,    kind=RP)
        enddo
     endif

  enddo

  write(IO_FID_LOG,*)
  write(IO_FID_LOG,*) '+++ pole fill start'

  call COMM_var( var(:,:,:,:), var_pl(:,:,:,:), ADM_kall, 4 )

  write(IO_FID_LOG,*) "##### (prc,rgnid) #####"
  do l  = 1, ADM_lall
  do k = ADM_kmin, ADM_kmin
     write(IO_FID_LOG,*)
     write(IO_FID_LOG,'(A9)',advance='no') "        |"
     do i = 1, ADM_gall_1d
        write(IO_FID_LOG,'(I9)',advance='no') i
     enddo
     write(IO_FID_LOG,*)

     do j = ADM_gall_1d, 1, -1
        write(IO_FID_LOG,'(I8,A1)',advance='no') j, "|"
        do i = 1, ADM_gall_1d
           ij = ADM_gall_1d * (j-1) + i
           write(IO_FID_LOG,'(A1,I3,A1,I3,A1)',advance='no') &
                      '(',int(var(ij,k,l,1)),',',int(var(ij,k,l,2)),')'
        enddo
        write(IO_FID_LOG,*)
     enddo
  enddo
  enddo

  do l  = 1, ADM_lall_pl
  do k = ADM_kmin, ADM_kmin
     write(IO_FID_LOG,*)
     write(IO_FID_LOG,'(A9)',advance='no') "        |"
     do i = 1, ADM_gall_pl
        write(IO_FID_LOG,'(I9)',advance='no') i
     enddo
     write(IO_FID_LOG,*)

     write(IO_FID_LOG,'(I8,A1)',advance='no') j, "|"
     do ij = 1, ADM_gall_pl
        write(IO_FID_LOG,'(A1,I3,A1,I3,A1)',advance='no') &
                   '(',int(var_pl(ij,k,l,1)),',',int(var_pl(ij,k,l,2)),')'
     enddo
     write(IO_FID_LOG,*)
  enddo
  enddo

  write(IO_FID_LOG,*) "##### (i,j) #####"
  do l  = 1, ADM_lall
  do k = ADM_kmin, ADM_kmin
     write(IO_FID_LOG,*)
     write(IO_FID_LOG,'(A9)',advance='no') "        |"
     do i = 1, ADM_gall_1d
        write(IO_FID_LOG,'(I9)',advance='no') i
     enddo
     write(IO_FID_LOG,*)

     do j = ADM_gall_1d, 1, -1
        write(IO_FID_LOG,'(I8,A1)',advance='no') j, "|"
        do i = 1, ADM_gall_1d
           ij = ADM_gall_1d * (j-1) + i
           write(IO_FID_LOG,'(A1,I3,A1,I3,A1)',advance='no') &
                      '(',int(var(ij,k,l,3)),',',int(var(ij,k,l,4)),')'
        enddo
        write(IO_FID_LOG,*)
     enddo
  enddo
  enddo

  do l  = 1, ADM_lall_pl
  do k = ADM_kmin, ADM_kmin
     write(IO_FID_LOG,*)
     write(IO_FID_LOG,'(A9)',advance='no') "        |"
     do i = 1, ADM_gall_pl
        write(IO_FID_LOG,'(I9)',advance='no') i
     enddo
     write(IO_FID_LOG,*)

     write(IO_FID_LOG,'(I8,A1)',advance='no') j, "|"
     do ij = 1, ADM_gall_pl
        write(IO_FID_LOG,'(A1,I3,A1,I3,A1)',advance='no') &
                   '(',int(var_pl(ij,k,l,3)),',',int(var_pl(ij,k,l,4)),')'
     enddo
     write(IO_FID_LOG,*)
  enddo
  enddo

  return
end subroutine COMM_debugtest
\end{LstF90}
%
This section does the test of \src{COMM_var}.
%
The procedure is the same with the case of \src{COMM_data_transfer},
using \src{prc}, \src{rgnid}, \src{i} and \src{j} as a test data,
writes out these values the before and after calling this subroutine.

See \autoref{s:comm_input_result} for example.




\subsection{Input data and result}\label{s:comm_input_result}

This kernel program doesn't need any input data, except configuring
namelist file \file{communication.cnf}.

Several configuration is prepared in \file{data} directory.
%
The default one is as follows.

\begin{LstLog}
################################################################################

&ADMPARAM
    glevel = 4,
    rlevel = 0,
    vlayer = 20,
    debug  = .true.,
/

&COMMPARAM
    debug    = .true.,
    testonly = .true.,
/

################################################################################
\end{LstLog}

Script files in \file{run} directory executes this kernel program with
two MPI processes.
Each process outputs logfile named \file{msg.pe000000} and
\file{msg.pe000001}.

These files are so long and skip here.
%
There are example output files in \file{reference} directory.
%
These outputs must be the same except white space changes.


\subsection{Sample of perfomance result}

Here's an example of the performance result part of the log output.
Below is an example executed with the machine environment described in \autoref{s:measuring_env}.
%
Note that in this program kernel part is iterated one time.

\begin{LstLog}
 *** Computational Time Report
 *** Rap level is            2
 *** ID=001 : MAIN_Kernel_ALL                  T=     0.020 N=        1
 *** ID=002 : MAIN_COMM_data_transfer          T=     0.001 N=        2
 *** ID=003 : MAIN_COMM_var                    T=     0.000 N=        1
\end{LstLog}
